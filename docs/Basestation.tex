\documentclass[12pt,english,twoside]{book}
\usepackage[T1]{fontenc}
\usepackage[latin1]{inputenc}
\usepackage{geometry}
\geometry{verbose,letterpaper,tmargin=1.25in,bmargin=1.25in,lmargin=1in,rmargin=1in}
\usepackage{fancyhdr}
\pagestyle{fancy}
\setcounter{secnumdepth}{3}
\setcounter{tocdepth}{3}
\usepackage{array}
\usepackage{float}
\usepackage{amsmath}

\makeatletter

%%%%%%%%%%%%%%%%%%%%%%%%%%%%%% LyX specific LaTeX commands.
%% Bold symbol macro for standard LaTeX users
\providecommand{\boldsymbol}[1]{\mbox{\boldmath $#1$}}

%% Because html converters don't know tabularnewline
\providecommand{\tabularnewline}{\\}

\usepackage{babel}
\makeatother
\begin{document}

\title{Basestation Documentation}


\author{APPLIED PHYSICS LABORATORY \\
 UNIVERSITY OF WASHINGTON}


\date{v2.00 Revised: April 18, 2007}

\maketitle
\tableofcontents{}

\chapter{Introduction}

This document describes the installation, operation and file formats that are
directly related to the Seaglider Basestation code.  There are both reference
and user manual components contained within this document.  Readers should be
familiar with the code Seaglider documentation set.

\chapter{Basestation Overview}

\chapter{Basestation Output Files}

\section{Introduction}

One of the main functions of the Seaglider basestation is to convert
the files received from the Seaglider to formats that are readable
and useful to the user. The current basestation software takes the
compressed Seaglider dive data and log files as described above, and
does the requisite processing to produce files with the following
naming convention.

\begin{quote}
\textbf{p instrument\_id dive\_number . {[}log, asc, eng, pvt]}
\end{quote}
This processing includes uncompressing and reassembling the transmitted
file chunks, and detection and correction of common transmission-induced
content errors. The basestation also handles an automated notification
system. The file processing and notification systems are invoked when
the Seaglider logs out, by the \textbf{.logout} facility of the shell.


\subsection{LOG file}

The format and contents of this file are the same as the log file on the
Seaglider compact flash as described in \textbf{Seaglider File Formats}
reference manual.


\subsection{ASC file}

The ASC, or ASCII, files are created on the basestation. They are essentially
the reconstituted (uncompressed, reassembled, and differentially summed)
versions of the data (DAT) files created on the Seaglider, and described in
\textbf{Seaglider File Formats} reference manual.


\subsubsection{Name}

The basestation calls this file

\textbf{p instrument\_id dive\_number . asc}.


\subsubsection{Format}

The file is divided into a \textit{header} section and a \textit{data}
section. The header consists of a series of lines of the format: \[
\textbf{tag}:\textbf{value}\textit{newline}\]
 Where \textbf{tag} is one of the tag values defined below and the
value is the tags value. The value field is interpreted based on the
tag. There may be leading whitespace between the : separator and the
actual value.

The header section terminates with the \textbf{data} tag. The balance
of the file is the data recorded during the dive.


\subsubsection{Header Tags}


\subsubsection{version tag}

This tag contains the version of the software glider that generated
this data file. The format for the version is \textbf{MAJOR}.\textbf{MINOR},
where MAJOR and MINOR are two digit integers.


\subsubsection{glider tag}

This tag contains the seagliders individual id value. The value is
between 1 and 999, with no leading zeros.


\subsubsection{mission tag}

This tag contains the mission id this glider was on when this file
was generated. The value maybe any positive integer, with no leading
zeros.


\subsubsection{dive tag}

This tag contains the specific dive the glider was on when it collected
the data. The value may be any positive integer, with no leading zeros.


\subsubsection{start tag}

This tag contains the starting time stamp for the dive specified above,
expressed in UTC. The format or the value is:

\begin{quote}
\textbf{mon}\space\textbf{day} \textbf{year} \textbf{hour} \textbf{min}
\textbf{sec} 
\end{quote}
The definitions for the the elements of this group are in Table \ref{StartTag}.

%
\begin{table}
\begin{centering}\begin{tabular}{|l|p{4.5in}|}
\hline 
\textbf{mon} &
Two digit number representing the month (JAN = 1) \tabularnewline
\hline 
\textbf{day} &
Two digit number representing day of the month \tabularnewline
\hline 
\textbf{year} &
Two or three digit number representing the year. For years 2000 or
earlier, the two digits are the last two digits of the year. For years
2001 to 2099, the leading digit is 1, and the next two digits are
the last two digits of the year. \tabularnewline
\hline 
\textbf{hour} &
Two digit number representing the hours (starting with 0) \tabularnewline
\hline 
\textbf{min} &
Two digit number representing the minutes (starting with zero) \tabularnewline
\hline 
\textbf{sec} &
Two digit number representing the seconds (starting with zero) \tabularnewline
\hline
\end{tabular}\par\end{centering}


\caption{Start tag format}

\label{StartTag} 
\end{table}


\textit{Note: no leading zeros in any of the above values}


\subsubsection{columns tag}

The value field for this tag contains a comma delimited list of strings
that indicate by their position what values are recorded in each of
the columns in the data section of the file. The first 10 columns
are alway in a fix order for any glider (Table \ref{FixedOrderColumns}.
The remaining columns (Table \ref{SensorColumns}) are determined
by what sensors are installed in what ports on the actual glider.

%
\begin{table}
\begin{centering}\begin{tabular}{|l|p{4.5in}|}
\hline 
\textbf{Header Tag} &
\textbf{Description} \tabularnewline
\hline 
\textbf{rec} &
Record number of the individual sample \tabularnewline
\hline 
\textbf{elaps\_t} &
Time since the start of the dive \tabularnewline
\hline 
\textbf{depth} &
Depth (cm) at the start of the sample \tabularnewline
\hline 
\textbf{heading} &
Vehicle heading at the start of the sample (degress Magnetic x 10) \tabularnewline
\hline 
\textbf{pitch} &
Vehicle pitch angle at the start of the sample (degress x 10, positve
upward) \tabularnewline
\hline 
\textbf{roll} &
Vehicle roll at the start of the sample (degress x 10, positive starbord
wing down \tabularnewline
\hline 
\textbf{AD\_pitch} &
Pitch mass position (A/D counts) \tabularnewline
\hline 
\textbf{AD\_roll} &
Roll mass position (A/D counts) \tabularnewline
\hline 
\textbf{AD\_vbd} &
VBD position (A/D counts) \tabularnewline
\hline 
\textbf{GC\_phase} &
Encoded GC Phase (see Table \ref{GCEncoding}) \tabularnewline
\hline
\end{tabular}\par\end{centering}


\caption{Fixed Order Columns}

\label{FixedOrderColumns} 
\end{table}


%
\begin{table}
\begin{centering}\begin{tabular}{|l|c|}
\hline 
\textbf{GC Encoding} &
\textbf{Description} \tabularnewline
\hline 
1 &
Pitch change \tabularnewline
\hline 
2 &
VBD change \tabularnewline
\hline 
3 &
Roll \tabularnewline
\hline 
4 &
Turning \tabularnewline
\hline 
5 &
Roll back (to center) \tabularnewline
\hline 
6 &
Passive mode \tabularnewline
\hline
\end{tabular}\par\end{centering}


\caption{GC Encodings}

\label{GCEncoding} 
\end{table}


%
\begin{table}
\begin{centering}\begin{tabular}{|l|p{4.5in}|}
\hline 
\textbf{Header Tag} &
\textbf{Description} \tabularnewline
\hline 
\textbf{TempFreq} &
Temperature (cycle counts of 4MHz in 255 cycles of signal frequency) \tabularnewline
\hline 
\textbf{CondFreq} &
Conductivity (cycle counts of 4MHz in 255 cycles of signal frequency) \tabularnewline
\hline 
\textbf{O2Freq} &
Oxygen concentration (cycle counts of 4Mhz in 255 cycles of signal
frequency)\tabularnewline
\hline 
\textbf{redRef} &
Red reference (A/D counts)\tabularnewline
\hline 
\textbf{redCount} &
Red backscatter (A/D counts)\tabularnewline
\hline 
\textbf{blueRef} &
Blue reference (A/D counts)\tabularnewline
\hline 
\textbf{blueCount} &
Blue backscatter (A/D counts)\tabularnewline
\hline 
\textbf{fluorCount} &
Fluorometer (A/D counts)\tabularnewline
\hline 
\textbf{VFtemp} &
BB2F temperature\tabularnewline
\hline 
\textbf{O2} &
Optional Aanderaa optode oxygen concentration\tabularnewline
\hline 
\textbf{temp} &
Optional Aanderaa optode temperature\tabularnewline
\hline 
\textbf{dphase} &
Optional Aanderaa optode dphase\tabularnewline
\hline
\end{tabular}\par\end{centering}


\caption{Sensor Columns}

\label{SensorColumns} 
\end{table}



\subsubsection{data tag}

The data tag has no value associated - it simple exists as a delimiter
between the header section and data.


\subsection{Data}

Data follows with one line per sample, terminated by a <newline>.
Empty fields, which indicate no sample was returned for that sensor,
are indicated by NaN. This could be because a sensor was not installed
for that deployment, or that the sensor was not enabled for that particular
sample, as controlled by the \textbf{science} file.


\subsection{Example}

\begin{quotation}
\noindent {\scriptsize version: 66.00}{\scriptsize \par}

\noindent {\scriptsize glider: 107}{\scriptsize \par}

\noindent {\scriptsize mission: 1}{\scriptsize \par}

\noindent {\scriptsize dive: 3}{\scriptsize \par}

\noindent {\scriptsize start: 5 10 106 20 59 20}{\scriptsize \par}

\noindent {\scriptsize columns: rec,elaps\_t,depth,heading,pitch,roll,AD\_pitch,AD\_roll,AD\_vbd,GC\_phase,TempFreq,CondFreq,O2Freq,redRef,redCount,}{\scriptsize \par}

\noindent {\scriptsize blueRef,blueCount,fluorCount,VFtemp,}{\scriptsize \par}

\noindent {\scriptsize data:}{\scriptsize \par}

\noindent {\scriptsize 0 19 76 3426 -721 -69 366 2028 1076 2 257049
179336 NaN NaN NaN NaN NaN NaN NaN }{\scriptsize \par}

\noindent {\scriptsize 1 24 83 3423 -727 -65 366 2028 1231 2 256853
179404 NaN NaN NaN NaN NaN NaN NaN }{\scriptsize \par}

\noindent {\scriptsize 2 29 77 3443 -732 -72 366 2028 1377 2 256640
179492 NaN NaN NaN NaN NaN NaN NaN }{\scriptsize \par}

\noindent {\scriptsize $\vdots$}{\scriptsize \par}
\end{quotation}

\section{ENG file}

The ENG, or engineering, files are created on the basestation. They
restate data contained in the ASC and LOG files, but with the Seaglider
control state and attitude observations converted into engineering
units.


\subsection{Name}

The basestation calls this file

\textbf{p instrument\_id dive\_number . eng}.


\subsection{Format}

The file is divided into a \textit{header} section and a \textit{data}
section. The header consists of a series of lines of the format: \[
\textbf{\% tag}:\textbf{value}\textit{newline}\]
 Where \textbf{tag} is one of the tag values defined below and the
value is the tags value. The value field is interpreted based on the
tag. There may be leading whitespace between the : separator and the
actual value.

The header section terminates with the \textbf{data} tag. The balance
of the file is the data recorded during the dive.


\subsection{Header Tags}


\subsubsection{version tag}

This tag contains the version of the software glider that generated
this data file. The format for the version is \textbf{MAJOR}.\textbf{MINOR},
where MAJOR and MINOR are two digit integers.


\subsubsection{glider tag}

This tag contains the seagliders individual id value. The value is
between 1 and 999, with no leading zeros.


\subsubsection{mission tag}

This tag contains the mission id this glider was on when this file
was generated. The value maybe any positive integer, with no leading
zeros.


\subsubsection{dive tag}

This tag contains the specific dive the glider was on when it collected
the data. The value may be any positive integer, with no leading zeros.


\subsubsection{start tag}

This tag contains the starting time stamp for the dive specified above,
expressed in UTC. The format or the value is:

\begin{quote}
\textbf{mon}\space\textbf{day} \textbf{year} \textbf{hour} \textbf{min}
\textbf{sec} 
\end{quote}
The definitions for the the elements of this group are in Table \ref{StartTag}.

%
\begin{table}
\begin{centering}\begin{tabular}{|l|p{4.5in}|}
\hline 
\textbf{mon} &
Two digit number representing the month (JAN = 1) \tabularnewline
\hline 
\textbf{day} &
Two digit number representing day of the month \tabularnewline
\hline 
\textbf{year} &
Two or three digit number representing the year. For years 2000 or
earlier, the two digits are the last two digits of the year. For years
2001 to 2099, the leading digit is 1, and the next two digits are
the last two digits of the year. \tabularnewline
\hline 
\textbf{hour} &
Two digit number representing the hours (starting with 0) \tabularnewline
\hline 
\textbf{min} &
Two digit number representing the minutes (starting with zero) \tabularnewline
\hline 
\textbf{sec} &
Two digit number representing the seconds (starting with zero) \tabularnewline
\hline
\end{tabular}\par\end{centering}


\caption{Start tag format}

\label{StartTag} 
\end{table}


\textit{Note: no leading zeros in any of the above values}


\subsubsection{columns tag}

The value field for this tag contains a comma delimited list of strings
that indicate by their position what values are recorded in each of
the columns in the data section of the file. The first 11 columns
are alway in a fix order for any glider (Table \ref{FixedOrderColumns}.
The remaining columns (Table \ref{SensorColumns}) are determined
by what sensors are installed in what ports on the actual glider.

%
\begin{table}
\begin{centering}\begin{tabular}{|l|p{4.5in}|}
\hline 
\textbf{Header Tag} &
\textbf{Description} \tabularnewline
\hline 
\textbf{elaps\_t\_0000}&
Time (s) since 0000UTC of the current day\tabularnewline
\hline 
\textbf{elaps\_t} &
Time (s) since the start of the dive \tabularnewline
\hline 
\textbf{condFreq}&
Conductivity frequency\tabularnewline
\hline 
\textbf{tempFreq}&
Temperature frequency\tabularnewline
\hline 
\textbf{depth} &
Depth (cm) at the start of the sample \tabularnewline
\hline 
\textbf{head}&
Vehicle heading at the start of the sample (degrees Magnetic) \tabularnewline
\hline 
\textbf{pitchAng}&
Vehicle pitch angle at the start of the sample (degrees positve upward) \tabularnewline
\hline 
\textbf{rollAng}&
Vehicle roll at the start of the sample (degrees, positive starboard
wing down) \tabularnewline
\hline 
\textbf{pitchCtl}&
Pitch mass position (cm relative to \textbf{\$C\_PITCH}, positive
nose up) \tabularnewline
\hline 
\textbf{rollCtl}&
Roll mass position (degrees relative to \textbf{\$C\_ROLL\_DIVE} or
\textbf{\$C\_ROLL\_CLIMB}, positive starboard wing down) \tabularnewline
\hline 
\textbf{vbcCC}&
VBD value (cc relative to \textbf{\$C\_VBD}, positive buoyant) \tabularnewline
\end{tabular}\par\end{centering}


\caption{Fixed Order Columns}

\label{FixedOrderColumns} 
\end{table}




%
\begin{table}
\begin{centering}\begin{tabular}{|l|p{4.5in}|}
\hline 
\textbf{Header Tag} &
\textbf{Description} \tabularnewline
\hline 
\textbf{O2Freq} &
Oxygen concentration (cycle counts of 4Mhz in 255 cycles of signal
frequency)\tabularnewline
\hline 
\textbf{redRef} &
Red reference (A/D counts)\tabularnewline
\hline 
\textbf{redCount} &
Red backscatter (A/D counts)\tabularnewline
\hline 
\textbf{blueRef} &
Blue reference (A/D counts)\tabularnewline
\hline 
\textbf{blueCount} &
Blue backscatter (A/D counts)\tabularnewline
\hline 
\textbf{fluorCount} &
Fluorometer (A/D counts)\tabularnewline
\hline 
\textbf{VFtemp} &
BB2F temperature\tabularnewline
\hline 
\textbf{O2} &
Optional Aanderaa optode oxygen concentration\tabularnewline
\hline 
\textbf{temp} &
Optional Aanderaa optode temperature\tabularnewline
\hline 
\textbf{dphase} &
Optional Aanderaa optode dphase\tabularnewline
\hline
\end{tabular}\par\end{centering}


\caption{Sensor Columns}

\label{SensorColumns} 
\end{table}



\subsubsection{data tag}

The data tag has no value associated - it simple exists as a delimiter
between the header section and data.


\subsection{Data}

Data follows with one line per sample, terminated by a <newline>.
Empty fields, which indicate no sample was returned for that sensor,
are indicated by NaN. This could be because a sensor was not installed
for that deployment, or that the sensor was not enabled for that particular
sample, as controlled by the \textbf{science} file.\newpage{}


\subsection{Example}

\begin{quotation}
{\tiny \%version:65.03}{\tiny \par}

{\tiny \%glider:107}{\tiny \par}

{\tiny \%mission:1}{\tiny \par}

{\tiny \%dive:2}{\tiny \par}

{\tiny \%start:5 10 106 20 6 7}{\tiny \par}

{\tiny \%columns:elaps\_t\_0000,elaps\_t,condFreq,tempFreq,depth,head,pitchAng,rollAng,pitchCtl,rollCtl,vbdCC,o2\_freq,redRef,redCount,}{\tiny \par}

{\tiny blueRef,blueCount,fluorCount,VFTemp}{\tiny \par}

{\tiny \%data: }{\tiny \par}

{\tiny 72386.0 19.0 5745.249 3966.217 79.000 344.5 -74.3 -2.8 -12.066
-0.17 476.6 NaN NaN NaN NaN NaN NaN NaN }{\tiny \par}

{\tiny 72391.0 24.0 5741.788 3957.093 76.000 344.9 -73.4 -5.7 -12.066
-0.17 438.4 NaN NaN NaN NaN NaN NaN NaN }{\tiny \par}

{\tiny 72396.0 29.0 5743.373 3957.354 64.000 343.0 -73.6 -6.9 -12.066
-0.17 402.0 NaN NaN NaN NaN NaN NaN NaN }{\tiny \par}

{\tiny $\vdots$}{\tiny \par}
\end{quotation}

\section{PVT files}

PVT, or private, files are created on the basestation. They contain
data that was originally in the logfile that could pose a security
problem if propagated off the basestation (as the logfile may well
be). Thus, the data is stripped from the log file and placed in the
matched pvt file.


\subsection{Format}

The PVT is very similar to a Logfile in format - it starts with a header that is
identical in layout to the one described in Section LogFile in the
\textbf{Seaglider File Formats} reference manual. Following the header is a
single section of parameters, as listed in Table \ref{PVTParameters}.

%
\begin{table}
\begin{centering}\begin{tabular}{|l|c|}
\hline 
\textbf{Parameter} &
\textbf{Description} \tabularnewline
\hline 
\textbf{\$PASSWD} &
See Parameter Reference\tabularnewline
\hline 
\textbf{\$TEL\_PREFIX} &
See Parameter Reference\tabularnewline
\hline 
\textbf{\$TEL\_NUM} &
See Parameter Reference\tabularnewline
\hline 
\textbf{\$ALT\_TEL\_PREFIX} &
See Parameter Reference\tabularnewline
\hline 
\textbf{\$ALT\_TEL\_NUM} &
See Parameter Reference\tabularnewline
\hline
\end{tabular}\par\end{centering}


\caption{PVT file parameters}

\label{PVTParameters} 
\end{table}



\section{comm.log}

The \textbf{comm.log} file is a log of all the communications sessions
by a particular Seaglider on a given deployment. It is appended during
each communications session. It is a plain-text file that resides
in the particular Seaglider's home directory. It captures the communications
session details and is useful as a monitoring and debugging tool.
It is useful to run \emph{tail -f comm.log} in the Seaglider's home
directory during (or while waiting for) communication sessions.


\section{convert.log}

The \textbf{convert.log} file logs the output from the \emph{convert.pl}
script that performs the data conversion and notification functions
of the basestation. It is appended during each invocation. This file
is the first place to look when debugging problems with the data conversion. 

\chapter{Basestation Input Files}

\section{Introduction}
The Basestation's operation is primarily controlled by configuration files that
are read each time the conversion processed is started.  

\section{.pagers file}

The \textbf{.pagers} file controls the automatic notification system.
It allows any of three types of messages to be sent to any valid email
address. This service is run by the data conversion script, which
is invoked by a glider logout or disconnection.

\subsection{Format}

The \textbf{.pagers} file is a line-oriented format. Comment lines
are indicated by a leading \textbf{\#}. The lines are of the following
format.

\begin{quote}
\textbf{email\_address,<service>{[}, service]{[},service] }
\end{quote}
where email\_address is any valid email address (user@mailhost), and
service may be any or all of the services shown in Table .

%
\begin{table}[H]

\caption{.pagers file services}

\begin{tabular}{|c|c|}
\hline 
\textbf{Service}&
\textbf{Description}\tabularnewline
\hline
\hline 
\textbf{gps}&
Most recent GPS position, and recovery code if in recovery\tabularnewline
\hline 
\textbf{alerts}&
Basestation data conversion alerts (see \textbf{convert.log})\tabularnewline
\hline 
\textbf{recov}&
Only if Seaglider in recovery, includes most recent GPS position and
recovery code\tabularnewline
\hline
\end{tabular}
\end{table}

\subsection{Example}
\begin{quotation}

\#Seaglider pilot

5555551234@mmode.com,gps,recov

chief\_pilot@seaglider.com,recov,alerts

\#Iridium Phone

881655512345@msg.iridium.com,gps
\end{quotation}

\section{.urls file}

The \textbf{.urls} file specifies URLs on which to run GET for each
processed dive. This can be used for any supported httpd function,
and is mainly used to poll for data transfers to support visualization
servers.

\subsection{Format}

The \textbf{.urls} file is a line-oriented format. Comments are indicated
by leading \textbf{\#}'s. The first entry on the line is the timeout
(in seconds) to wait for a response to the GET. The basestation script
\emph{convert.pl} adds arguments 'instrument\_name=sg<nnn>\&dive=<dive>'
with the proper separator.


\subsection{Example}
\begin{quotation}
\# These are URLs to GET for each processed dive

10 http://www.seaglider.com/\textasciitilde{}glider/cgi-bin/update.cgi
\end{quotation}

\section{.mailer file}

The \textbf{.mailer} file controls the automatic data delivery system via email.
It allows any or all of the output files that the Basestation creates during
processing to be sent as email attachments to any email address. This service is
run by the data conversion script, which is invoked by a glider logout or
disconnection.

\subsection{Format}

The \textbf{.mailer} file is a line-oriented format. Comment lines
are indicated by a leading \textbf{\#}. The lines are of the following
format.

\begin{quote}
\textbf{email\_address,{[}format,]all|<filetype>{[}{[},filetype],...]}
\end{quote}
where \textbf{email\_address} is any valid SMTP email address (user@mailhost),
\textbf{format} may be not specified, in which case, each file is sent as a
unique attachment, or is \textbf{gzip}, in which case all files are placed in a
gziped tar file attachment, \textbf{all}, indicating all file types should be
sent and \textbf{filetype} may be any or all of the services shown in Table .

%
\begin{table}[H]

\caption{.mailer file types}

\begin{tabular}{|l|p{3.5in}|}
\hline 
\textbf{Filetype}&
\textbf{Description}\tabularnewline
\hline
\hline 
\textbf{comm}&
Send the comm.log file.  Note: this file is always compressed using gzip, 
regardless of the format setting.\tabularnewline
\hline 
\textbf{eng|log| pro|bpo|csv|asc|cap|nc}&
Send files whose extension match those specified.
\tabularnewline
\hline
\end{tabular}
\end{table}

\subsection{Example}
\begin{quotation}
\#Field team on board ship

science@shipboard.org,gzip,log,eng

\#Pilot working remotely

pilot@u.washington.edu,all

\end{quotation}
\end{document}
